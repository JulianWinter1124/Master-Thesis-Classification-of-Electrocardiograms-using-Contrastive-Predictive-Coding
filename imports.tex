

%%%%%%%%%%%%%%%%%%%%%%%%%%%%%%%%%%%%%%%%%%%%%%%%%%%%%%%%%%%
% Obere Titelmakros. Editieren Sie diese Datei nur, wenn
% Sie sich ABSOLUT sicher sind, was Sie da tun!!!
% (Z.B. zum Abaendern der BA-Vorlage in eine MA-Vorlage)
% Uni Duesseldorf
% Lehrstuhl fuer Datenbanken und Informationssysteme
% Version 2.2 - 2.3.2010
%%%%%%%%%%%%%%%%%%%%%%%%%%%%%%%%%%%%%%%%%%%%%%%%%%%%%%%%%%%
\newcommand{\AN}{twoside}
\newcommand{\AUS}{}


%\newcommand{\englisch}{}
%\newcommand{\deutsch}{\usepackage[german]{babel}}

%% Die folgenden auskommentierten Optionen dienen der automatischen
%% Erkennung des Latex-Kompilers und dem Setzen der davon abhängigen
%% Einstellungen. Bei Problem z.B. mit dem Einbinden von verschiedenen
%% Grafiktypen bei Verwendung von PdfLatex oder Latex, einfach die
%% verschiedenen \usepackage(s) ausprobieren. (Mit diesen Einstellungen
%% funktionierte diese Vorlage bei der Verwenundg von latex.exe als
%% Kompiler bei den meisten Studierenden.)

%\newif\ifpdf \ifx\pdfoutput\undefined
%\pdffalse % we are not running pdflatex
%\else
%\pdfoutput=1 % we are running pdflatex
%\pdfcompresslevel=9 % compression level for text and image;
%\pdftrue \fi

\documentclass[11pt,a4paper, \zweiseitig]{article}
\usepackage{ifthen}


\usepackage{amsmath,amssymb}
%\usepackage[iso]{umlaute}
\usepackage[utf8]{inputenc}
\usepackage{palatino} % palatino Schriftart
%\usepackage{makeidx} % um ein Index zu erstellen
\usepackage[nottoc]{tocbibind}
\usepackage[T1]{fontenc} %fuer richtige Trennung bei Umlauten
\usepackage{fancybox} % fuer die Rahmen
\usepackage{shortvrb}
\usepackage{url}
\usepackage[table]{xcolor}
\usepackage[font=footnotesize, labelfont=bf]{subcaption}
\usepackage{graphicx}%[draft]
\usepackage{courier}
\usepackage{forest}
\usepackage{pdfpages}
%\usepackage{standalone}
%\usepackage{hyperref}
\usepackage[]{algorithm2e}
\newcommand\mycommfont[1]{\footnotesize\ttfamily\color{codegreen}{#1}}
%\SetCommentSty{mycommfont}
\usepackage{booktabs}
\usepackage{csquotes}
\usepackage{longtable}
\usepackage{tabularx}
\usepackage{adjustbox}
\usepackage{soul}
\sethlcolor{lightgray}
\usepackage{float}
\usepackage{enumitem}
\usepackage{multirow}
\usepackage[font=footnotesize, labelfont=bf]{caption}
\captionsetup{font=footnotesize}
\captionsetup[figure]{font=footnotesize}
\captionsetup[table]{font=footnotesize}

\usepackage{tcolorbox}
\usepackage{listings}
\usepackage{xcolor}
%TODO:printer
%\pdfimageresolution=300 
%\usepackage{titletoc}
%\usepackage[loadonly]{titlesec}
%TODO: printer
\usepackage[colorlinks,citecolor=black,linkcolor=black]{hyperref}
%\usepackage[colorlinks,citecolor=blue,linkcolor=red]{hyperref} %anklickbares Inhaltsverzeichnis
\DeclareMathSymbol{*}{\mathbin}{symbols}{"01}
\DeclareMathOperator*{\mean}{mean}
\DeclareMathOperator*{\std}{std}

\def\chapterautorefname{Chapter}%
\def\sectionautorefname{Section}%
\def\subsectionautorefname{Subsection}%
\def\subsubsectionautorefname{Subsubsection}%
\def\paragraphautorefname{Paragraph}%
\def\tableautorefname{Table}%
\def\equationautorefname{Equation}%

\newlength\mystoreparindent
\newenvironment{myparindent}[1]{%
	\setlength{\mystoreparindent}{\the\parindent}
	\setlength{\parindent}{#1}
}{%
	\setlength{\parindent}{\mystoreparindent}
}

\definecolor{codegreen}{rgb}{0,0.6,0}
\definecolor{codegray}{rgb}{0.5,0.5,0.5}
\definecolor{codepurple}{rgb}{0.58,0,0.82}
\definecolor{backcolour}{rgb}{0.95,0.95,0.92}

\lstdefinestyle{mystyle}{
	backgroundcolor=\color{backcolour},   
	commentstyle=\color{codegreen},
	keywordstyle=\color{magenta},
	numberstyle=\tiny\color{codegray},
	stringstyle=\color{codepurple},
	basicstyle=\ttfamily\footnotesize,
	breakatwhitespace=false,         
	breaklines=true,                 
	captionpos=b,                    
	keepspaces=true,                 
	numbers=left,                    
	numbersep=5pt,                  
	showspaces=false,                
	showstringspaces=false,
	showtabs=false,                  
	tabsize=3
}

\lstset{
	style=mystyle,
	literate=
	{á}{{\'a}}1 {é}{{\'e}}1 {í}{{\'i}}1 {ó}{{\'o}}1 {ú}{{\'u}}1
	{Á}{{\'A}}1 {É}{{\'E}}1 {Í}{{\'I}}1 {Ó}{{\'O}}1 {Ú}{{\'U}}1
	{à}{{\`a}}1 {è}{{\`e}}1 {ì}{{\`i}}1 {ò}{{\`o}}1 {ù}{{\`u}}1
	{À}{{\`A}}1 {È}{{\'E}}1 {Ì}{{\`I}}1 {Ò}{{\`O}}1 {Ù}{{\`U}}1
	{ä}{{\"a}}1 {ë}{{\"e}}1 {ï}{{\"i}}1 {ö}{{\"o}}1 {ü}{{\"u}}1
	{Ä}{{\"A}}1 {Ë}{{\"E}}1 {Ï}{{\"I}}1 {Ö}{{\"O}}1 {Ü}{{\"U}}1
	{â}{{\^a}}1 {ê}{{\^e}}1 {î}{{\^i}}1 {ô}{{\^o}}1 {û}{{\^u}}1
	{Â}{{\^A}}1 {Ê}{{\^E}}1 {Î}{{\^I}}1 {Ô}{{\^O}}1 {Û}{{\^U}}1
	{ã}{{\~a}}1 {ẽ}{{\~e}}1 {ĩ}{{\~i}}1 {õ}{{\~o}}1 {ũ}{{\~u}}1
	{Ã}{{\~A}}1 {Ẽ}{{\~E}}1 {Ĩ}{{\~I}}1 {Õ}{{\~O}}1 {Ũ}{{\~U}}1
	{œ}{{\oe}}1 {Œ}{{\OE}}1 {æ}{{\ae}}1 {Æ}{{\AE}}1 {ß}{{\ss}}1
	{ű}{{\H{u}}}1 {Ű}{{\H{U}}}1 {ő}{{\H{o}}}1 {Ő}{{\H{O}}}1
	{ç}{{\c c}}1 {Ç}{{\c C}}1 {ø}{{\o}}1 {å}{{\r a}}1 {Å}{{\r A}}1
}

\ifthenelse{\boolean{\biber}}{
  % only needed for biber
  \usepackage[style=numeric,natbib=true,backend=biber,mincitenames=1,maxcitenames=2,maxbibnames=99,uniquelist=true, sorting=none]{biblatex}%style=authoryear,dashed=false

  % https://tex.stackexchange.com/a/334703/8850
  \AtEveryBibitem{%
    \clearfield{issn}
    \clearfield{isbn}
    \clearfield{doi}
    \clearfield{location}
    \clearlist{location}
    \clearlist{address}

    \ifentrytype{online}{}{% Remove url except for @online
      \clearfield{url}
    }
  }
}
{}%no else

% Falls es bei \citet ein Komma zwischen Name und Jahr gibt:
% https://tex.stackexchange.com/questions/312539/unwanted-comma-between-author-and-year-using-citet-command
% (thx @ Markus Brenneis)
%\DeclareDelimFormat[cbx@textcite]{nameyeardelim}{\addspace}



\ifthenelse{\equal{\sprache}{deutsch}}{
  \usepackage[ngerman]{babel}
  % Bibtex u.a -> et al.
  \ifthenelse{\boolean{\biber}}{
    \DefineBibliographyStrings{ngerman}{
      andothers = {{et\,al\adddot}},
    }
    \newcommand{\references}{Literatur}
  }
  {} % do nothing when not using biber
  \usepackage[autostyle, german=quotes]{csquotes} % Deutsche Anführungszeichen im Literaturverzeichnis (thx @ Markus Brenneis)

}{ \newcommand{\references}{References}
\usepackage[english]{babel}
}

\usepackage{a4wide} % ganze A4 Weite verwenden



%\ifpdf
%\usepackage[pdftex,xdvi]{graphicx}
%\usepackage{thumbpdf} %thumbs fuer Pdf
%\usepackage[pdfstartview=FitV]{hyperref} %anklickbares Inhaltsverzeichnis
%\else
%\usepackage[dvips,xdvi]{graphicx}
\usepackage{amsmath}

\newcommand{\code}[1]{\texttt{#1}}

%\fi

\newcommand{\redt}[1] {
  \textcolor{red}{#1}}

\DeclareRobustCommand{\bbone}{\text{\usefont{U}{bbold}{m}{n}1}}
\DeclareMathOperator{\EX}{\mathbb{E}} %expected value

\newcommand{\oranget}[1] {
  \textcolor{orange}{#1}}

\newcommand{\purplet}[1] {
  \textcolor{purple}{#1}}

%%%%%%%%%%%%%%%%%%%%%%% Massangaben fuer die Arbeit %%%%%%%%%%%%%%%
\setlength{\textwidth}{15cm}

\setlength{\oddsidemargin}{35mm}
\setlength{\evensidemargin}{25mm}

\addtolength{\oddsidemargin}{-1in}
\addtolength{\evensidemargin}{-1in}

\ifthenelse{\boolean{\biber}}{\addbibresource{references.bib}}{}

%\makeindex